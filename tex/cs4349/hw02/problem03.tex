\section[Problem 3]{Show that $k \lg k = \Theta(n)$ implies $k = \Theta\left(\frac{n}{\ln n}\right)$.}
If $f_n = \Theta(g_n)$, then $\Theta(f_n) = g_n$ by symmetric property of big-$\Theta$. We can use this with some algebra to solve this problem:
\begin{align*}
	k \ln k &= \Theta(n) \Longrightarrow \Theta(k \ln k) = n \\
	\ln [n] &= \Theta ( \ln [k \ln k ] ) \\
	&=
	\Theta (\ln k + \ln \ln k) \\
	&= \Theta(\ln k) \\
	n &= \Theta(k \ln k) \\
	\frac{n}{\ln n} &= \frac{\Theta(k \ln k)}{\Theta(\ln k)} = \Theta \left( \frac{k \ln k}{\ln k} \right) = \Theta(k) \\
	\Theta(k) &= \frac{n}{\ln n} \\ 
	k &= \Theta \left( \frac{n}{\ln n} \right) 
\end{align*}
