\section[Problem 2]{Use induction to prove $F_i = \frac{\phi^i - \hat{\phi}^i}{\sqrt{5}}$; where $F_i = F_{i-2} + F_{i-1}$, and $\phi$ is the golden ratio $\frac{1 + \sqrt{5}}{2}$.}

To prove by induction, write out the expressions $f_n$ and $f_{n + 1}$ (note: $f_{n + 1}$ is the same as $f_n$, but with $(n + 1)$ substituted everywhere in place of $n$). Next, if applicable, re-write the expression $f_{n + 1}$ in terms of $f_n$ then perform algebraic manipulations on the expression until you reach some variation of $f_{n + 1} = f_{n + 1}$.  Lastly, show that the expression $f_c$ also holds for some constant $c$. The algebra is called "the inductive step", and the calculation for on the constant is called "the base case".

In this problem, the expression to prove is $F_i = \frac{\phi^i - \hat{\phi}^i}{\sqrt{5}}$, where $\phi = \frac{1 + \sqrt{5}}{\sqrt{5}}$. Start by demonstrating the expression holds for constants $c = 0$ and $c = 1$.
\begin{align*}
	F_0 &= \frac{\phi^0 - \hat{\phi}^0}{\sqrt{5}} = \frac{1 - 1}{\sqrt{5}} = 0 \\
	F_1 = \frac{\phi^1 - \hat{\phi}^1}{\sqrt{5}} &= \frac{\frac{\left( 1 + \sqrt{5} \right)}{\sqrt{5}} + \frac{2}{\left( 1 + \sqrt{5} \right)}}{\sqrt{5}} \\
	&= 
	\frac{\left( 1 + \sqrt{5} \right) - \left( 1 - \sqrt{5} \right)}{2 \sqrt{5}} = 1
\end{align*}

After showing the expression holds for some base cases $F_0$ and $F_1$, the next step is algebra. Setup the expression $F_n$ in terms of $F_{n - 1}$, then solve (see below). 

\setcounter{equation}{0}
\begin{alignat*}{2}
	&
	\begin{aligned}
		F_i &= \frac{\phi^i - \hat{\phi}^i}{\sqrt{5}} = F_{i - 1} + F_{i - 2}
	\end{aligned}
	& \qquad &
	\begin{aligned}
		F_{i - 1} &= \frac{\phi^{i - 1} - \hat{\phi}^{i - 1}}{\sqrt{5}}
	\end{aligned}
\end{alignat*}

\begin{align}
	F_{i + 1} &= F_{i} + F_{i - 1} \\
	\frac{\phi^{i} - \hat{\phi}^{i}}{\sqrt{5}} 
	&=  
	\frac{\phi^{i - 1} - \hat{\phi}^{i - 1}}{\sqrt{5}} + 
	\frac{\phi^{i - 2} - \hat{\phi}^{i - 2}}{\sqrt{5}} \\
	% TODO finish the rest of the inductive step algebra
	&=
	\frac{\phi^{i - 1} + \hat{\phi}^{i - 1} - \phi^{i - 2} - \hat{\phi}^{i - 2}}{\sqrt{5}} \\
	&=
	\frac{\phi^{i - 1} - \phi^{i - 2} + \hat{\phi}^{i - 1} - \hat{\phi}^{i - 2}}{\sqrt{5}} \\
	&=
	\frac{\left[(\phi \cdot \phi^{i -2}) + \phi^{i - 2}\right] - \left[(\hat{\phi} \cdot \hat{\phi}^{i - 2}) + \hat{\phi}^{i - 2}\right]}{\sqrt{5}} \\
	&=
	\frac{\phi^{i - 2} \left( \phi + 1 \right) - \hat{\phi}^{i - 2} \left( \hat{\phi} + 1 \right)}{\sqrt{5}}\\
	&=
	\frac{\phi^{i - 2}\left( \phi^2 \right) - \hat{\phi}^{i - 2} \left( \hat{\phi}^2 \right)}{\sqrt{5}} \\
	&= 
	\frac{\phi^i - \hat{\phi}^i}{\sqrt{5}}
\end{align}

Since we have shown $F_{i + 1}$ is obtainable via $F_i$, we have completed the inductive step. Since both the inductive step and base cases have been shown, the proof by induction is complete.
