\section[Problem 2]{Use induction to prove $F_i = \frac{\phi^i - \hat{\phi}^i}{\sqrt{5}}$; where $F_i = F_{i-2} + F_{i-1}$, and $\phi$ is the golden ratio $\frac{1 + \sqrt{5}}{2}$.}

To prove by induction, write out the expressions $f_n$ and $f_{n + 1}$ (note: $f_{n + 1}$ is the same as $f_n$, but with $(n + 1)$ substituted everywhere in place of $n$). Next, if applicable, re-write the expression $f_{n + 1}$ in terms of $f_n$ then perform algebraic manipulations on the expression until you reach some variation of $f_{n + 1} = f_{n + 1}$.  Lastly, show that the expression $f_c$ also holds for some constant $c$. The algebra is called "the inductive step", and the calculation for on the constant is called "the base case".

In this problem, the expression to prove is $F_i = \frac{\phi^i - \hat{\phi}^i}{\sqrt{5}}$, where $\phi = \frac{1 + \sqrt{5}}{\sqrt{5}}$. Start by demonstrating the expression holds for constants $c = 0, c = 1$ (e.g., the "base case").

\begin{align}
	F_0 &= \frac{\phi^0 - \hat{\phi}^0}{\sqrt{5}} \\
		&= 0
\end{align}
