\section[Problem 6]{Order the following functions such that $f_1 = \Omega(f_2), f_2 = \Omega(f_3), ..., f_{29} = \Omega(f_{30})$, and partition them into equivalence classes such that each function is big-$\Theta$ of each other.}

Note that, in terms of growth, $f_1 = \Omega(f_2)$ means $f_1 \leq f_2$. Therefore, the order of functions $f_1 = \Omega(f_2), f_2 = \Omega(f_3), ..., f_{29} = \Omega(f_{30})$ is as follows:
$2^{2^n} = \Omega \left((n + 1)! \right), $
$(n + 1)! = \Omega \left(n! \right), $
$n! = \Omega \left(e^n \right), $
$e^n = \Omega \left(n \cdot 2^n \right), $
$n \cdot 2^n = \Omega \left(2^n \right), $
$2^n = \Omega \left(\left( \frac{3}{2} \right)^n \right), $
$\left( \frac{3}{2} \right)^n = \Omega \left(n^{\lg \lg n} \right), $
$n^{\lg \lg n} = \Omega \left(\left( \lg n \right)^{\lg n} \right), $
$\left( \lg n \right)^{\lg n} = \Omega \left((\lg n)! \right), $
$(\lg n)! = \Omega \left(N^3 \right), $
$N^3 = \Omega \left(n^2 \right), $
$n^2 = \Omega \left(4^{\lg n} \right), $
$4^{\lg n} = \Omega \left(\lg (n!) \right), $
$\lg (n!)  = \Omega \left(n \lg n \right), $
$n \lg n = \Omega \left(2^{\lg n} \right), $
$2^{\lg n} = \Omega \left(n \right), $
$n = \Omega \left(\left( \sqrt{2} \right)^{\lg n} \right), $
$\left( \sqrt{2} \right)^{\lg n} = \Omega \left(\sqrt{n} \right), $
$\sqrt{n} = \Omega \left(2^{\sqrt{2 \lg n}} \right), $
$2^{\sqrt{2 \lg n}} = \Omega \left(\lg ^2 n \right), $
$\lg ^2 n = \Omega \left(\ln n \right), $
$\ln n = \Omega \left(\sqrt{\lg n} \right), $
$\sqrt{\lg n} = \Omega \left(\ln \ln n \right), $
$\ln \ln n = \Omega \left(2^{\lg ^* n} \right), $
$2^{\lg ^* n} = \Omega \left(\lg ^* n \right), $
$\lg ^* n = \Omega \left(\lg * (\lg n) \right), $
$\lg * (\lg n) = \Omega \left(\lg (\lg * n) \right), $
$\lg (\lg * n) = \Omega \left(n^{\frac{1}{\lg n}} \right), $
$n^{\frac{1}{\lg n}} = \Omega (1)$. See notes on \co{1.5} for how to calculate whether a function $f_n = \Theta(g_n)$.

An equivalence class is a set containing elements that all adhere to some property. In this case, the elements are functions $f$, and the property is that each function is big-$\Theta$ of every other function in the set. The functions above can be partitioned into the following equivalence classes:
$\left\{ n^{\lg \lg n}, \left( \lg n \right)^{\lg n} \right\}$, 
$\left\{ n^2, 4^{\lg n} \right\}$,
$\left\{ \lg(n!), n \lg n \right\}$,
$\left\{ 2^{\lg n}, n \right\}$,
$\left\{ \left( \sqrt{2} \right) ^{\lg n}, \sqrt{n} \right\}$, \\ % goes into the right margin here
$\left\{ \lg^* n, \lg^* (\lg n) \right\}$,
$\left\{ n^{\frac{1}{\lg n}}, 1 \right\}$.

