\chapter{Assignment 2}


% problem 1
\section[Problem 1]{Are either $\lceil \lg n \rceil!$ or $\lceil \lg \lg n \rceil!$ polynomially bounded?}
Polynomially bounded means $f_n = O(n^k)$ for some constant $k$ (e.g., whether $f_n \leq c \cdot n^k$ for constants $c$ and $k$ as $n$ approaches $\infty$). For the first function $\lceil \lg n \rceil!$, without loss of generality, assume $n = 2^a$ (where $a \in \mathbb{N}$). 
\begin{align*}
	\lceil \lg n \rceil ! & \leq c \cdot n^k \\
	\lg (2^a) ! & \leq c \cdot (2^a)^k \\
	a! & \leq c \cdot 2^{ak}
\end{align*}

The statement $a! \leq c \cdot 2^{ak}$ is a contradiction, as the factorial function $a!$ is not exponentially bounded. Therefore, $\lceil \lg n \rceil!$ is not polynomially bounded (via proof by contradiction). For the second function $\lceil \lg \lg n \rceil!$, without loss of generality, assume $n = 2^{2^a}$ (where $a \in \mathbb{N}$).
\begin{align*}
	 \lceil \lg \lg n \rceil! & \leq c \cdot n^k \\
	 \lg \lg \left( 2^{2^a} \right) ! & \leq c \cdot \left( 2^{2^a} \right)^k \\
	 a! & \leq c \cdot 2^{k \cdot {2^a}} \\
	 1 \cdot 2 \cdot 3 \cdots a & \leq c \cdot \left( 2^{2k} \cdot 2^{4k} \cdot 2^{8k} \cdots 2^{2^a \cdot k} \right)
\end{align*}

The statement $1 \cdot 2 \cdot 3 \cdots a \leq c \cdot \left( 2^{2k} \cdot 2^{4k} \cdot 2^{8k} \cdots 2^{2^a k} \right)$ is obviously true. Therefore $\lceil \lg \lg n \rceil!$ is polynomially bounded.



% problem 2
\section[Problem 2]{Use induction to prove $F_i = \frac{\phi^i - \hat{\phi}^i}{\sqrt{5}}$; where $F_i = F_{i-2} + F_{i-1}$, and $\phi$ is the golden ratio $\frac{1 + \sqrt{5}}{2}$.}

\section[Problem 3]{Show that $k \lg k = \Theta(n)$ implies $k = \Theta\left(\frac{n}{n \ln n}\right)$.}

% problem 3
\section[Problem 4]{Are either $2^{n + 1}$ or $2^{2n}$ big-$O$ of $2^n$?}

% problem 4
\section[Problem 5]{For each pair of functions $(A, B)$, indicate whether $A$ is $O, o, \Omega, \omega$, or $\Theta$ of $B$. Assume $k \geq 1$, $\epsilon > 0$, $c > 1$ are constants.}

\begin{center}
	\begin{tabular}{cc|c|c|c|c|c}
		$A$ & $B$ & $O$ & $o$ & $\Omega$ & $\omega$ & $\Theta$ \\ \hline
		$\lg^k n$ & $n^{\epsilon}$ & yes & yes & yes & yes & yes \\ \hline
		$n^k$ & $c^n$ & yes & yes & yes & yes & yes \\ \hline
		$\sqrt{n}$ & $n^{\sin n}$ & yes & yes & yes & yes & yes \\ \hline
		$2^n$ & $2^{n/2}$ & yes & yes & yes & yes & yes \\ \hline
		$n^{\lg c}$ & $c^{\lg n}$ & yes & yes & yes & yes & yes \\ \hline
		$\lg(n!)$ & $\lg(n^n)$ & yes & yes & yes & yes & yes \\ \hline
		$A$ & $B$ & yes & yes & yes & yes & yes
	\end{tabular}
\end{center}

\section[Problem 6]{Order the following functions such that $f_1 = \Omega(f_2), f_2 = \Omega(f_3), ..., f_{29} = \Omega(f_{30})$, and partition them into equivalence classes such that each function is big-$\Theta$ of each other.}
