\documentclass[twoside,12pt,a4paper,english]{memoir} 
\usepackage[a4paper,vmargin=84pt,hmargin=108pt,footskip=48pt]{geometry}
% latex article configuration, by @zachleach
% 2024-02-10

% set global font size
\documentclass[12pt]{article}		 

% configure page margins
\usepackage[vmargin=84pt,hmargin=108pt,footskip=48pt]{geometry}

% set font 
\usepackage{fouriernc}				

% set american font encoding and hyphenation
\usepackage[T1]{fontenc}			
\usepackage[USenglish]{babel}	

% configure how far section numbering ("label") hangs in left margin
\usepackage{titlesec}				
\newlength\sectionindent
\setlength\sectionindent{54pt}

% configure \titleformat{\section}{}{}{}{}{}
\titleformat{\section}				
{\normalfont\Large}{\llap{\makebox[\sectionindent][l]{\thesection\hfill}}}{0pt}{}	
\titlespacing*{\section}{0pt}{0pt}{12pt}

% configure \titleformat{\subsection}{}{}{}{}{}
\titleformat{\subsection}			
{\normalfont}{\llap{\makebox[\sectionindent][l]{\thesubsection\hfill}}}{0pt}{}
\titlespacing*{\subsection}{0pt}{0pt}{12pt}

% configue enumerated lists
\usepackage[shortlabels]{enumitem}	
\setlist[enumerate]{left=-4pt, labelsep=7pt, topsep=4pt, parsep=0pt, before={\vspace{0pt}}}

% configure bulleted lists
\setlist[itemize]{label=$\bullet$, left=1pt, labelsep=11pt, itemsep=0pt, topsep=2pt}

% enable syntax color definitions
\usepackage{xcolor}					
\definecolor{islamicgreen}{HTML}{00BF00}
\definecolor{irisblue}{HTML}{00BFBF}
\definecolor{deepmagenta}{HTML}{BF00BF}
\definecolor{codegreen}{rgb}{0,0.6,0}

% enable code blocks ("listings")
\usepackage{listings}				

% set codeblock border horizontal margins
\lstset{xleftmargin=4pt,xrightmargin=4pt}

% set codeblock vertical spacing
\lstset{aboveskip=8pt,belowskip=8pt}

% define C++ code style	 '\begin{lstlisting}[style=cpp]'
\lstdefinestyle{cpp} {
	language=C++,
	frame=single,
	basicstyle=\ttfamily\small,
	showstringspaces=false,
	columns=flexible,
	commentstyle=\color{codegreen},
	keywordstyle=\color{blue},
	stringstyle=\color{codegreen}
}

% define txt code style	 '\begin{lstlisting}[style=txt]'
\lstdefinestyle{txt} {
	frame=single,
	basicstyle=\ttfamily\small,
	showstringspaces=false,
	columns=flexible
    commentstyle=\color{black},
	keywordstyle=\color{black},
    stringstyle=\color{black}
}

% enable image insertion '\includegraphics[width=\textwidth]{image.png}'
\usepackage{graphicx}

% enable math blocks '\begin{align}'
\usepackage{amsmath}

% configure math block ('align' environment) vertical spacing between lines
\setlength{\jot}{12pt}
\setlength{\abovedisplayskip}{0pt}
\setlength{\belowdisplayskip}{0pt}

% enable underline, define underline, bold, italic, and code font (e.g., \ul, \bb, \ii, \co, respectively)
\usepackage{soul}
\newcommand{\bb}[1] {\textbf{#1}}
\newcommand{\ii}[1] {\textit{#1}}
\newcommand{\co}[1] {\texttt{#1}}


\begin{document}
\chapter{Assignment 2}
% problem 1
\section[Problem 1]{Are either $\lceil \lg n \rceil!$ or $\lceil \lg \lg n \rceil!$ polynomially bounded?}

\ul{Polynomially bounded means $f_n = O(n^k)$} for some constant $k$ (e.g., whether $f_n \leq c \cdot n^k$ for constants $c$ and $k$ as $n$ approaches $\infty$). 

\noindent
For the first function $\lceil \lg n \rceil!$, without loss of generality, assume $n = 2^a$ (where $a \in \mathbb{N}$). 
The statement $a! \leq c \cdot 2^{ak}$ (see below) is a contradiction, as the factorial function $a!$ is not exponentially bounded. Therefore, $\lceil \lg n \rceil!$ is not polynomially bounded (via proof by contradiction).
\begin{align*}
	\lceil \lg n \rceil ! & \leq c \cdot n^k \\
	\lg (2^a) ! & \leq c \cdot (2^a)^k \\
	a! & \leq c \cdot 2^{ak}
\end{align*}


For the second function $\lceil \lg \lg n \rceil!$, without loss of generality, assume $n = 2^{2^a}$ (where $a \in \mathbb{N}$).
The statement $1 \cdot 2 \cdot 3 \cdots a \leq c \cdot \left( 2^{2k} \cdot 2^{4k} \cdot 2^{8k} \cdots 2^{2^a k} \right)$ (see below) is obviously true. Therefore $\lceil \lg \lg n \rceil!$ is polynomially bounded (via direct proof).
\begin{align*}
	 \lceil \lg \lg n \rceil! & \leq c \cdot n^k \\
	 \lg \lg \left( 2^{2^a} \right) ! & \leq c \cdot \left( 2^{2^a} \right)^k \\
	 a! & \leq c \cdot 2^{k \cdot {2^a}} \\
	 1 \cdot 2 \cdot 3 \cdots a & \leq c \cdot \left( 2^{2k} \cdot 2^{4k} \cdot 2^{8k} \cdots 2^{2^a \cdot k} \right)
\end{align*}


 
\section[Problem 2]{Use induction to prove $F_i = \frac{\phi^i - \hat{\phi}^i}{\sqrt{5}}$; where $F_i = F_{i-2} + F_{i-1}$, and $\phi$ is the golden ratio $\frac{1 + \sqrt{5}}{2}$.}

To prove by induction, write out the expressions $f_n$ and $f_{n + 1}$ (note: $f_{n + 1}$ is the same as $f_n$, but with $(n + 1)$ substituted everywhere in place of $n$). Next, if applicable, re-write the expression $f_{n + 1}$ in terms of $f_n$ then perform algebraic manipulations on the expression until you reach some variation of $f_{n + 1} = f_{n + 1}$.  Lastly, show that the expression $f_c$ also holds for some constant $c$. The algebra is called "the inductive step", and the calculation for on the constant is called "the base case".

In this problem, the expression to prove is $F_i = \frac{\phi^i - \hat{\phi}^i}{\sqrt{5}}$, where $\phi = \frac{1 + \sqrt{5}}{\sqrt{5}}$. Start by demonstrating the expression holds for constants $c = 0, c = 1$ (e.g., the "base case"). 
\begin{align}
	F_0 &= \frac{\phi^0 - \hat{\phi}^0}{\sqrt{5}} = \frac{1 - 1}{\sqrt{5}} \\
		&= 0 
\end{align}

After showing the expression holds for some base cases $F_0$ and $F_1$, the next step is algebra. Setup the expression $F_{n + 1}$ in terms of $F_n$, then solve (see below).

\setcounter{equation}{0}
\begin{alignat*}{2}
	&
	\begin{aligned}
		F_i &= \frac{\phi^i - \hat{\phi}^i}{\sqrt{5}} = F_{i - 1} + F_{i - 2}
	\end{aligned}
	& \qquad &
	\begin{aligned}
		F_{i + 1} &= \frac{\phi^{i + 1} - \hat{\phi}^{i + 1}}{\sqrt{5}} = F_{i} + F_{i - 1}
	\end{aligned}
\end{alignat*}

\begin{align}
	F_{i + 1} &= F_{i} + F_{i - 1} \\
	\frac{\phi^{i + 1} - \hat{\phi}^{i + 1}}{\sqrt{5}} 
	&=  
	\frac{\phi^{i} - \hat{\phi}^{i}}{\sqrt{5}} +
	\frac{\phi^{i - 1} - \hat{\phi}^{i - 1}}{\sqrt{5}} \\
	&=
	\frac{\phi^{i} - \hat{\phi}^{i}}{\sqrt{5}} +
	\frac{\phi^{i - 1} - \hat{\phi}^{i - 1}}{\sqrt{5}}
\end{align}



%\section[Problem 3]{Show that $k \lg k = \Theta(n)$ implies $k = \Theta\left(\frac{n}{\ln n}\right)$.}
If $f_n = \Theta(g_n)$, then $\Theta(f_n) = g_n$ by symmetric property of big-$\Theta$. We can use this with some algebra to solve this problem:
\begin{align*}
	k \ln k &= \Theta(n) \Longrightarrow \Theta(k \ln k) = n \\
	\ln [n] &= \Theta ( \ln [k \ln k ] ) \\
	&=
	\Theta (\ln k + \ln \ln k) \\
	&= \Theta(\ln k) \\
	n &= \Theta(k \ln k) \\
	\frac{n}{\ln n} &= \frac{\Theta(k \ln k)}{\Theta(\ln k)} = \Theta \left( \frac{k \ln k}{\ln k} \right) = \Theta(k) \\
	\Theta(k) &= \frac{n}{\ln n} \\ 
	k &= \Theta \left( \frac{n}{\ln n} \right) 
\end{align*}

%\section[Problem 4]{Are either $2^{n + 1}$ or $2^{2n}$ big-$O$ of $2^n$?}

The former is, the latter is not. 

Suppose $2^{n + 1} = O(2^n)$, then $2 \cdot 2^n \leq c \cdot 2^n$ as $n \rightarrow \infty$; this is obviously true for all $c \geq 2$.  Regards to whether $2^{2n} = O(2^n)$, consider the inequality $2^{2n} \leq c \cdot 2^n$. This is equivalent to saying $2^n \cdot 2^n \leq c \cdot 2^n$. Dividing both sides by $2^n$ gives $2^n \leq c$, which is obviously false.

%\section[Problem 5]{For each pair of functions $(A, B)$, indicate whether $A$ is $O, o, \Omega, \omega$, or $\Theta$ of $B$. Assume $k \geq 1$, $\epsilon > 0$, $c > 1$ are constants.}

In terms of growth,
\ul{$f_n = \Omega(g_n)$ means	$f_n \geq c \cdot g_n$},
\ul{$f_n = \omega(g_n)$ means	$f_n > c \cdot g_n$},
\ul{$f_n = O(g_n)$ means		$f_n \leq c \cdot g_n$},
\ul{$f_n = o(g_n)$ means		$f_n < c \cdot g_n$}, and
\ul{$f_n = \Theta(g_n)$ means	$f_n == c \cdot g_n$}. 

\begin{center}
	\begin{tabular}{lcc|c|c|c|c|c}
		& $A$ & $B$ & $O$ & $o$ & $\Omega$ & $\omega$ & $\Theta$ \\ \hline
		a. & $\lg^k n$ & $n^{\epsilon}$ & yes & yes &  &  &  \\ \hline
		b. & $n^k$ & $c^n$				& yes & yes &  &  &  \\ \hline
		c. & $\sqrt{n}$ & $n^{\sin n}$	&  &  &  &  &  \\ \hline
		d. & $2^n$ & $2^{n/2}$			&  &  & yes & yes &  \\ \hline
		e. & $n^{\lg c}$ & $c^{\lg n}$	& yes &  & yes &  & yes \\ \hline
		f. & $\lg(n!)$ & $\lg(n^n)$		& yes &  & yes &  & yes 
	\end{tabular}
\end{center}

\ul{To demonstrate whether something is big-something of something, isolate the constant $c$ in the equaility and observe whether it holds.} Also note that big-$\Omega$ precludes little-$o$ and big-$O$ precludes little-$\omega$ (e.g., if $f_n = O(g_n)$, then $f_n = \omega(g_n)$ is false, and vice-versa).
 done
%\section[Problem 6]{Order the following functions such that $f_1 = \Omega(f_2), f_2 = \Omega(f_3), ..., f_{29} = \Omega(f_{30})$, and partition them into equivalence classes such that each function is big-$\Theta$ of each other.}

Note that, in terms of growth, $f_1 = \Omega(f_2)$ means $f_1 \leq f_2$. Therefore, the order of functions $f_1 = \Omega(f_2), f_2 = \Omega(f_3), ..., f_{29} = \Omega(f_{30})$ is as follows:
$2^{2^{n + 1}} = \Omega \left(2^{2^n} \right), $
$2^{2^n} = \Omega \left((n + 1)! \right), $
$(n + 1)! = \Omega \left(n! \right), $
$n! = \Omega \left(e^n \right), $
$e^n = \Omega \left(n \cdot 2^n \right), $
$n \cdot 2^n = \Omega \left(2^n \right), $
$2^n = \Omega \left(\left( \frac{3}{2} \right)^n \right), $
$\left( \frac{3}{2} \right)^n = \Omega \left(n^{\lg \lg n} \right), $
$n^{\lg \lg n} = \Omega \left(\left( \lg n \right)^{\lg n} \right), $
$\left( \lg n \right)^{\lg n} = \Omega \left((\lg n)! \right), $
$(\lg n)! = \Omega \left(N^3 \right), $
$N^3 = \Omega \left(n^2 \right), $
$n^2 = \Omega \left(4^{\lg n} \right), $
$4^{\lg n} = \Omega \left(\lg (n!) \right), $
$\lg (n!)  = \Omega \left(n \lg n \right), $
$n \lg n = \Omega \left(2^{\lg n} \right), $
$2^{\lg n} = \Omega \left(n \right), $
$n = \Omega \left(\left( \sqrt{2} \right)^{\lg n} \right), $
$\left( \sqrt{2} \right)^{\lg n} = \Omega \left(\sqrt{n} \right), $
$\sqrt{n} = \Omega \left(2^{\sqrt{2 \lg n}} \right), $
$2^{\sqrt{2 \lg n}} = \Omega \left(\lg ^2 n \right), $
$\lg ^2 n = \Omega \left(\ln n \right), $
$\ln n = \Omega \left(\sqrt{\lg n} \right), $
$\sqrt{\lg n} = \Omega \left(\ln \ln n \right), $
$\ln \ln n = \Omega \left(2^{\lg ^* n} \right), $
$2^{\lg ^* n} = \Omega \left(\lg ^* n \right), $
$\lg ^* n = \Omega \left(\lg * (\lg n) \right), $
$\lg * (\lg n) = \Omega \left(\lg (\lg * n) \right), $
$\lg (\lg * n) = \Omega \left(n^{\frac{1}{\lg n}} \right), $
$n^{\frac{1}{\lg n}} = \Omega (1)$. See notes from problem 5 for how to calculate whether a function $f_n = \Theta(g_n)$.

An equivalence class is a set containing elements that all adhere to some property. In this case, the elements are functions $f$, and the property is that each function is big-$\Theta$ of every other function in the set. The functions above can be partitioned into the following equivalence classes:
$\left\{ n^{\lg \lg n}, \left( \lg n \right)^{\lg n} \right\}$, 
$\left\{ n^2, 4^{\lg n} \right\}$,
$\left\{ \lg(n!), n \lg n \right\}$,
$\left\{ 2^{\lg n}, n \right\}$,
$\left\{ \left( \sqrt{2} \right) ^{\lg n}, \sqrt{n} \right\}$, \\ % goes into the right margin here
$\left\{ \lg^* n, \lg^* (\lg n) \right\}$,
$\left\{ n^{\frac{1}{\lg n}}, 1 \right\}$.

 done


\end{document}
