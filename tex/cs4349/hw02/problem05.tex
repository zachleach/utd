\section[Problem 5]{For each pair of functions $(A, B)$, indicate whether $A$ is $O, o, \Omega, \omega$, or $\Theta$ of $B$. Assume $k \geq 1$, $\epsilon > 0$, $c > 1$ are constants.}
\begin{center}
	\begin{tabular}{lcc|c|c|c|c|c}
		& $A$ & $B$ & $O$ & $o$ & $\Omega$ & $\omega$ & $\Theta$ \\ \hline
		a. & $\lg^k n$ & $n^{\epsilon}$ & yes & yes &  &  &  \\ \hline
		b. & $n^k$ & $c^n$				& yes & yes &  &  &  \\ \hline
		c. & $\sqrt{n}$ & $n^{\sin n}$	&  &  &  &  &  \\ \hline
		d. & $2^n$ & $2^{n/2}$			&  &  & yes & yes &  \\ \hline
		e. & $n^{\lg c}$ & $c^{\lg n}$	& yes &  & yes &  & yes \\ \hline
		f. & $\lg(n!)$ & $\lg(n^n)$		& yes &  & yes &  & yes 
	\end{tabular}
\end{center}

The main idea here is that (in terms of growth rate), 
$f_n = \Omega(g_n)$ means	$f_n \geq c \cdot g_n$,
$f_n = \omega(g_n)$ means	$f_n > c \cdot g_n$,
$f_n = O(g_n)$ means		$f_n \leq c \cdot g_n$,
$f_n = o(g_n)$ means		$f_n < c \cdot g_n$, and
$f_n = \Theta(g_n)$ means	$f_n == c \cdot g_n$. 

To demonstrate whether something is big-something of something, you perform algebraic manipulations on the respective aforementioned inequalities (e.g., isolate the constant $c$) and observe whether the inequality holds. \ul{Also note that big-$\Omega$ precludes little-$o$ and big-$O$ precludes little-$\omega$} (e.g., if $f_n = O(g_n)$, then $f_n = \omega(g_n)$ is false, and vice-versa).
