% body.tex
% 2024.02.13, by @zachleach

\section*{\Huge\bfseries Chapter 1}

\section{What is an algorithm}
An algorithm is a sequence of computational steps that transforms an input into an output, generally to solve a well-defined computational problem.

\section{What is a data structure}
Data structures are a way to store and organize data in memory to facilitate efficient access and modification (e.g., to enhance the speed of an algorithm).

\section{How to quantitatively measure algorithm efficiency}
Intuitively, it takes $c$ units of time to perform a given computational operation. Typically the number of operations required by an algorithm corresponds to the size of the input $n$; therefore, algorithmic efficiency is expressed as a function of input size. 

For instance, to sort $n$ integers in increasing order, the \ii{insertion sort} algorithm takes $c \cdot n^2$ units time, whereas the \ii{merge sort} takes $c \cdot n \lg n$. Comparably speaking then, the $n \lg n$ algorithm will outperform the $n^2$ algorithm for large input sizes $n$. 

There's an entire mathematical notation for identifying and comparing these input-efficiency functions for algorithms, called \ii{asymptotic notation}; it's discussed at length in chapter 3. Observe following input sizes $n$ which could be completed in time $t$ for each efficiency function $f_n$:

\begin{center}
	\begin{tabular}{c|c|c|c|c|c|c|c}
		& 1 second & 1 minute & 1 hour & 1 day & 1 month & 1 year & 1 century  \\ \hline
		$\lg n$		& $2^{10^6}$ & $2^{10^7}$ & $2^{10^9}$  & $2^{10^{10}}$ & $2^{10^{12}}$ & $2^{10^{13}}$ & $2^{10^{15}}$ \\ \hline
		$\sqrt{n}$	& $10^{12}$ & $10^{15}$ & $10^{19}$ & $10^{21}$ & $10^{24}$ & $10^{26}$ & $10^{30}$ \\ \hline
		$n$			& $10^{6}$ & $10^{7}$ & $10^{9}$ & $10^{10}$ & $10^{12}$ & $10^{13}$ & $10^{15}$ \\ \hline
		$n \lg n$	& $10^{4}$ & $10^{6}$ & $10^{8}$ & $10^{9}$ & $10^{10}$ & $10^{11}$ & $10^{13}$ \\ \hline
		$n^2$		& $10^{3}$ & $10^{3}$ & $10^{4}$ & $10^{5}$ & $10^{6}$ & $10^{6}$ & $10^{7}$ \\ \hline
		$n^3$		& 100 & 400 & 1500 & 4000 & $10^{4}$ & $10^{4}$ & $10^{5}$ \\ \hline
		$2^n$		& 20 & 25 & 30 & 35 & 40 & 45 & 50 \\ \hline
		$n!$		& 9 & 11 & 12 & 13 & 15 & 16 & 17
	\end{tabular}
\end{center}
