% body.tex
% 2024.02.20, by @zachleach

\section{Data Rate Problem}
It is desired to send a sequence of computer screen images over optical fiber. The screen is 3840 $\times$ 2160 pixels, each pixel being 24 bits. There are 60 screen images per second. What data rate is needed?

\begin{align*}
	\text{Data Rate} = \frac{\text{Number of bits}}{\text{Bits per second}}
\end{align*}
\vspace{4pt}

There are $24 \text{ bits} \cdot (3840 \times 2160)$ = 199,065,600 bits per image. Transmitting 60 images per second gives a data rate of data rate is $60 \cdot $ 199,065,600 = \ul{$1.194 \cdot 10^{10}$ bits per second.}

\section{FDM Multiplexing Problem}
Ten signals, each requiring 4000 Hz, are multiplexed onto a single channel using FDM. What is the minimum bandwidth required for the multiplexed channel? Assume that the guard bands are 400 Hz wide.

\begin{align*}
	\text{Bandwidth} = [\text{\# of channels} \cdot \text{channel bandwidth}] + [(\text{\# of channels} - 1) \cdot \text{guard band width}]
\end{align*}
\vspace{4pt}

Therefore, bandwidth = $[10 \cdot 4000 \text{Hz}] + [(9) \cdot 400 \text{Hz}]$ = 43,600 Hz.

\section{}
\section{}
\section{}
\section{}
\section{}
